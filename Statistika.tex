\documentclass[11pt]{article}
\usepackage[utf8]{inputenc}
\usepackage[slovene]{babel}
\usepackage{amsthm,amsfonts,amsmath,amssymb,url}
\usepackage{mathtools}
\usepackage{bm}
\usepackage{esvect}
\usepackage{bbm}


\textheight 210 true mm
\textwidth 146 true mm
\voffset=-17mm
\hoffset=-13mm

\newtheorem{Izrek}{{\sc Izrek}}[section]
\newtheorem{Trditev}[Izrek]{{\sc Trditev}}
\newtheorem{Posledica}[Izrek]{{\sc Posledica}}
\newtheorem{Definicija}[Izrek]{{\sc Definicija}}
\newtheorem{Zgled}[Izrek]{{\sc Zgled}}
\newtheorem{Opomba}[Izrek]{{\sc Opomba}}
\def\theIzrek{{\rm \arabic{section}.\arabic{Izrek}}}

\newenvironment{izrek}{\begin{Izrek}\sl}{\end{Izrek}}
\newenvironment{trditev}{\begin{Trditev}\sl}{\end{Trditev}}
\newenvironment{posledica}{\begin{Posledica}\sl}{\end{Posledica}}
\newenvironment{definicija}{\begin{Definicija}\rm }{\end{Definicija}}
\newenvironment{zgled}{\begin{Zgled}\rm }{\end{Zgled}}
\newenvironment{opomba}{\begin{Opomba}\rm }{\end{Opomba}}

\newenvironment{dokaz}[1][{\sc Dokaz}]{\begin{proof}[#1]\renewcommand*{\qedsymbol}{\(\blacksquare\)}}{\end{proof}}

\newcommand{\Mod}[1]{\hbox{ (mod } #1)}
\renewcommand\labelenumi{(\theenumi)}

\begin{document}
	
	\thispagestyle{empty}
	\begin{center}
		\begin{Large}
			{\bf Zapiski pri predmetu Statistika}
		\end{Large}
		
	\end{center}
	Minimalni katalog znanja, ki ga bom sproti dopolnjeval. Verjetno bom izpustil kakšen dokaz in pa kakšen zgled.
	\vfill
	\begin{center}
		Ljubljana, 2017 $\quad \quad $ Gregor Vavdi
	\end{center}
	\newpage
	\setcounter{page}{1}
	
	%%%%%%%%%%%%%%%%%%%%%%%%%%%%%%%%%%%%%%%%%%%Uvod%%%%%%%%%%%%%%%%%%%%%%%%%%%%%%%%%%
	\section{Motivacija}
	Kako bi "ocenili" verjetnost, da pri metu kovanca pade cifra?
	\\
	Izvedemo $n$ neodvisnih "enakih" (v istih razmerah, na enak način, pošteno oz.naključno) metov kovanca in iskano verjetnost ocenimo z razmerjem $\frac{\textnormal{število cifer}}{n}$.
	\\
	\\
	Igramo igro, kjer kroglico položimo v eno od treh škatel. Zmešamo škatle med seboj in poskušamo uganiti kje je kroglica. Če uganemo dobimo $10$, v nasprotnem primeru pa izgubimo $6$.
	\\
	\\
	Kako bi ocenili pričakovano vrednost te igre?
	\\
	Izvedemo $n$ neodvisnih slučajnih iger in pričakovano vrednost ene igre ocenimo z $\frac{\textnormal{skupni izkupiček}}{n}$.
	\\
	\\
	Zdi se nam, da mora z večjim vzorcem priti boljša ocena.
	\\
	\\
	V 18. stoletju je grof Buffon kovanec vrgel 4040-krat in dobil 2048 cifer. Ocenjena verjetnost cifre je $0.50689$.
	\\
	V 19. stoletju je Pason vrgel kovanec 12000-krat in dobil 6019 cifer. Ocenjena vrejetnost je $0.5016$.
	\\
	\\
	Aksiome verjetnosti zgradimo tako, da so naša mnenja glede vprašanj upravičena.
	
	\section{Konvergenca slučajnih spremenljivk in limitni izrek}
	\begin{Definicija}
		Naj bo $X_1,X_2,X_3,\ldots $ slučajne spremenljivke, definirane na skupnem prostoru $\Omega$.
		\begin{enumerate}
			\item
			Pravimo, da zaporedje $\{X_n\}_n$ konvergira k $X$ v porazdelitvi, če $$\lim\limits_{n \to \infty}{P(X_n \le x)} = P(X\le x)$$
			za vsa tista realna števila $x$, v katerih je komulativna porazdelitvena funkcija slučajne spremenljivke $X$ zvezna.
			\item
			Pravimo, da je $\{X_n\}_n$ konvergira k $X$ v verjetnosti, če velja:
			$$\lim\limits_{n \to \infty}{|X_n - X|>\varepsilon} = 0$$
			za vsak $\varepsilon > 0$.
			\item
			Pravimo, da $\{X_n\}_n$ k $X$ skoraj gotovo, če je:
			$$P(\{\omega \in \Omega | \exists \lim\limits_{n \to \infty}{X_n(\omega) = X(\omega)}\}) = 1$$	
		\end{enumerate}
	\end{Definicija}
	\begin{Trditev}
		Iz konvergence "skoraj gotovo" sledi konvergenca v verjetnosti.
	\end{Trditev}
	\begin{Trditev}
		(Neenakost Markova)
		\\
		Naj bo $X$ slučajna spremenljivka s pričakovano vrednostjo in $a>0$ pozitivna konstanta. Tedaj je:
		$$P(|X| \ge a) \le \frac{E[|X|]}{a}$$
	\end{Trditev}
	\begin{dokaz}
		Naj bo $a>0$. Pišemo $A =  \{|X| \ge a\} = \{\omega |\quad |X(\omega)| \ge a  \}$. Tedaj $|X| \ge a \cdot \mathcal{U}_A$. Sledi $E[|X|] \ge a \cdot P(A)$.
	\end{dokaz}
	\begin{Posledica}
		Naj bo $X$ slučajna spremenljivka s (končno) disperzijo. Tedaj velja
		$$P(|X - EX]|\ge \varepsilon) \le \frac{D(X)}{\varepsilon ^2}$$
		za vsako pozitivno število $\varepsilon$.
	\end{Posledica}
	\begin{dokaz}
		$$P(|X- E[X]| \ge \varepsilon) = P( (|X - E[X]|)^2 \le \varepsilon ^2) < \frac{E((X - E[X])^2)}{\varepsilon ^2} = \frac{D(X)}{\varepsilon ^2}$$
	\end{dokaz}
	\begin{Izrek}
	(Šibki zakon velikih števil)
	\\
	Naj bodo $X_1, X_2, \ldots \quad \Omega \to \mathbb{R}$ neodvisne in enako porazdeljene slučajne spremenljivke s pričakovano vrednostjo $\mu$ in (končnim) odklonom $\sigma$. Tedaj zaporedje "vzorčnih povprečij" $\frac{X_1,\ldots , X_n}{n}$ konvergira v verjetnosti h konstantni $\mu$.
	\end{Izrek}
	\begin{dokaz}
		Trdimo, da velja $\lim\limits_{n \to \infty}{P(|\frac{X_1,X_2,\ldots , X_n}{n}} - \mu| \ge \varepsilon) = 0$ za vsak pozitiven $\varepsilon > 0$. Pišimo $\bar{X} = \frac{X_1,\ldots , X_n}{n}$.
		$$P(|\bar{X} - \mu | > \varepsilon) \le P(|\bar{X} - \mu | \ge \varepsilon) \le \frac{D(\bar{X})}{\varepsilon ^2} = \frac{D(\frac{X_1,\ldots , X_n}{n})}{\varepsilon ^2} = \frac{1}{n^2 \varepsilon ^2}D(X_1) + \ldots + D(X_n) = \frac{\sigma ^2 }{n \varepsilon ^2}$$
		Sledi, da rezultat konvergira proti $0$, ko gre $n$ v neskončnost.
	\end{dokaz}
	\end{document}